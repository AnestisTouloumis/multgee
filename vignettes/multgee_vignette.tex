\documentclass[
]{jss}

\usepackage[utf8]{inputenc}

\providecommand{\tightlist}{%
  \setlength{\itemsep}{0pt}\setlength{\parskip}{0pt}}

\author{
Anestis Touloumis\\University of Brighton
}
\title{\proglang{R} Package \pkg{multgee}: A Generalized Estimating
Equations Solver for Multinomial Responses}

\Plainauthor{Anestis Touloumis}
\Plaintitle{R Package multgee: A Generalized Estimating Equations Solver
for Multinomial Responses}
\Shorttitle{\pkg{multgee}: GEE for Multinomial Responses}

\Abstract{
This vignette of the \proglang{R} package \pkg{multgee} is an updated
version of \citet{Touloumis2015}, published in the Journal of
Statistical Software. To cite \pkg{multgee} in publications, please use
\citet{Touloumis2015}. To cite the GEE methodology implemented in
\pkg{multgee}, please use \citet{Touloumis2012}.

The \proglang{R} package \pkg{multgee} implements the local odds ratios
generalized estimating equations (GEE) approach proposed by
\citet{Touloumis2012}, a GEE approach for correlated multinomial
responses that circumvents theoretical and practical limitations of the
GEE method. A main strength of \pkg{multgee} is that it provides GEE
routines for both ordinal (\code{ordLORgee}) and nominal
(\code{nomLORgee}) responses, while relevant software in \proglang{R}
and \proglang{SAS} are restricted to ordinal responses under a marginal
cumulative link model specification. In addition, \pkg{multgee} offers a
marginal adjacent categories logit model for ordinal responses and a
marginal baseline category logit model for nominal. Further, utility
functions are available to ease the local odds ratios structure
selection (\code{intrinsic.pars}), to constract confidence intervals
(\code{confint}), to extract the variance-covariance matrix of the
regression parameters (\code{vcov}) and to perform a Wald--type
goodness-of-fit test between two nested GEE models (\code{waldts}). We
demonstrate the application of \pkg{multgee} through a clinical trial
with clustered ordinal multinomial responses.
}

\Keywords{generalized estimating equations, nominal and ordinal
multinomial responses, local odds ratios, \proglang{R}}
\Plainkeywords{generalized estimating equations, nominal and ordinal
multinomial responses, local odds ratios, R}

%% publication information
%% \Volume{50}
%% \Issue{9}
%% \Month{June}
%% \Year{2012}
%% \Submitdate{}
%% \Acceptdate{2012-06-04}

\Address{
    Anestis Touloumis\\
  University of Brighton\\
  School of Computing, Engineering and Mathematics

Moulsecoomb, Brighton, BN2 4GJ, UK\\
  E-mail: \email{A.Touloumis@brighton.ac.uk}\\
  
  }


% Pandoc header

\usepackage{amsmath}     \usepackage{amsbsy}

\Volume{64} \Issue{8} \Month{March} \Year{2015} \Submitdate{2013-06-08}
\Acceptdate{2014-07-24} \DOI{10.18637/jss.v064.i08}

\begin{document}

\hypertarget{introduction}{%
\section{Introduction}\label{introduction}}

In several studies, the interest lies in drawing inference about the
regression parameters of a marginal model for correlated, repeated or
clustered multinomial variables with ordinal or nominal response
categories while the association structure between the dependent
responses is of secondary importance. The lack of a convenient
multivariate distribution for multinomial responses and the sensitivity
of ordinary maximum likelihood methods to misspecification of the
association structure led researchers to modify the GEE method of
\citet{Liang1986} in order to account for multinomial responses
\citep{Miller1993, Lipsitz1994, Williamson1995, Lumley1996, Heagerty1996, Parsons2006}.
These GEE approaches estimate the marginal regression parameter vector
by solving the same set of estimating equations as in \citet{Liang1986},
but differ in the way they parametrize and/or estimate
\(\boldsymbol \alpha\), a parameter vector that is usually defined to
describe a ``working'' assumption about the association structure.

\citet{Touloumis2012} showed that the joint existence of the estimated
marginal regression parameter vector and \(\hat{\boldsymbol \alpha}\)
cannot be assured in existing approaches. This is because the parametric
space of the proposed parameterizations of the association structure
depends on the marginal model specification even in the simple case of
bivariate multinomial responses. To address this issue,
\citet{Touloumis2012} defined \(\boldsymbol \alpha\) as a ``nuisance''
parameter vector that contains the marginalized local odds ratios
structure, that is the local odds ratios as if no covariates were
recorded, and they employed the family of association models
\citep{Goodman1985} to develop parsimonious and meaningful structures
regardless of the response scale. The practical advantage of the local
odds ratios GEE approach is that it is applicable to both ordinal and
nominal multinomial responses without being restricted by the marginal
model specification. Simulations in \citet{Touloumis2012} imply that the
local odds ratios GEE approach captures a significant portion of the
underlying correlation structure, and compared to the independence
``working'' model (i.e., assuming no correlation structure in the GEE
methodology), simple local odds ratios structures can substantially
increase the efficiency gains in estimating the regression vector of the
marginal model. Note that low convergence rates for the GEE approach of
\citet{Lipsitz1994} and \citet{Heagerty1996} did not allow the authors
to compare these approaches with the local odds ratios GEE approach
while the GEE approach of \citet{Parsons2006} was excluded from the
simulation design because its use is restricted to a cumulative logit
marginal model specification.

The \proglang{R} \citep{RCoreTeam2013} package \pkg{multgee} implements
the local odds ratios GEE approach and it is available from the
Comprehensive \proglang{R} Archive Network at
\url{https://CRAN.R-project.org/package=multgee}. To emphasize the
importance of reflecting the nature of the response scale on the
marginal model specification and on the marginalized local odds ratios
structure, two core functions are available in \pkg{multgee}:
\code{nomLORgee} which is appropriate for GEE analysis of nominal
multinomial responses and \code{ordLORgee} which is appropriate for
ordinal multinomial responses. In particular, options for the marginal
model specification include a baseline category logit model for nominal
response categories and a cumulative link model or an adjacent
categories logit model for ordinal response categories. In addition,
there are five utility functions that enable the user to: i) Perform
goodness-of-fit tests between two nested GEE models (\code{waldts}), ii)
construct Wald type confidence intervals (\code{confint}), iii) select
the local odds ratios structure based on the rule of thumb discussed in
\citet{Touloumis2012} (\code{intrinsic.pars}), iv) obtain the estimated
covariance matrix of the parameter estimates (\code{vcov}), and v)
construct a probability table (to be passed in the core functions) that
satisfies a desired local odds ratios structure (\code{matrixLOR}).

To appreciate the features of \pkg{multgee}, we briefly review GEE
software for multinomial responses in \proglang{SAS} \citep{SAS} and
\proglang{R}. The current version of \proglang{SAS} supports only the
independence ``working'' model under a marginal cumulative probit or
logit model for ordinal multinomial responses. To the best of our
knowledge, \proglang{SAS} macros \citep{Williamson1998, Yu2004}
implementing the approach of \citet{Williamson1995} are not publicly
available. The \proglang{R} package \pkg{repolr} \citep{Parsons2013}
implements the approach of \citet{Parsons2006} but it is restricted to
using a cumulative logit model. Another option for ordinal responses is
the function \code{ordgee} in the \proglang{R} package \pkg{geepack}
\citep{Hojsgaard2006}. This function implements the GEE approach of
\citet{Heagerty1996} but it seems to produce unreliable results for
multinomial responses. To illustrate this, we simulated independent
multinomial responses under a cumulative probit model specification with
a single time-stationary covariate for each subject and we employed
\code{ordgee} to obtain the GEE estimates from the independence
``working'' model. Description of the generative process can be found in
Scenario 1 of \citet{Touloumis2012} except that we used the values
\(-3,-1,1\) and \(3\) for the four category specific intercepts in order
to make the problem more evident. Based on \(1000\) simulation runs when
the sample size \(N=500\), we found that the bias of the GEE estimate of
\(\beta=1\) was \(\approx 4.8 \times 10^{28}\), indicating the presence
of a bug or -at least- of numerical problems for some situations.
Similar problems occurred for the alternative global odds ratios
structures in \code{ordgee}. In contrast to existing software,
\pkg{multgee} offers greater variety of GEE models for ordinal
responses, implements a GEE model for nominal responses and is not
limited to the independence ``working'' model, which might lead to
significant efficiency losses. Further, one can assess the goodness of
fit for two or more nested GEE models.

This paper is organized as follows. In Section
\protect\hyperlink{GEENotation}{2}, we present the theoretical
background of the local odds ratios GEE approach that is necessary for
the use of \pkg{multgee}. We introduce the marginal models implemented
in \pkg{multgee}, the estimation procedure for the ``nuisance''
parameter vector \(\boldsymbol \alpha\) and the asymptotic theory on
which GEE inference is based. We describe the arguments of the core GEE
functions (\code{nomLORgee}, \code{ordLORgee}) in Section
\protect\hyperlink{Description1}{3} while the utility functions
(\code{waldts}, \code{confint}, \code{intrinsic.pars}, \code{vcov},
\code{matrixLOR}) are described in Section
\protect\hyperlink{Description2}{4}. In Section
\protect\hyperlink{Example}{5}, we illustrate the use of \pkg{multgee}
in a longitudinal study with correlated ordinal multinomial responses.
We summarize the features of the package and provide a few practical
guidelines in Section \protect\hyperlink{Summary}{6}.

\hypertarget{GEENotation}{%
\section{Local odds ratios GEE approach}\label{GEENotation}}

For notational ease, suppose the data arise from a longitudinal study
with no missing observations. However, note that the local odds ratios
GEE approach is not limited neither to longitudinal studies nor to
balanced designs, under the strong assumption that missing observations
are missing completely at random \citep{Rubin1976}.

Let \(Y_{it}\) be the multinomial response for subject \(i\)
\((i=1,\ldots,N)\) at time \(t\) \((t=1,\ldots,T)\) that takes values in
\(\{1,2,\ldots,J\}\), \(J>2\). Define the response vector for subject
\(i\)
\[\mathbf {Y}_{i}=(Y_{it1},\ldots,Y_{i1(J-1)},Y_{i21},\ldots,Y_{i2(J-1)},\ldots,Y_{iT1},\ldots,Y_{iT(J-1)})^{\top},\]
where \(Y_{itj}=1\) if the response for subject \(i\) at time \(t\)
falls at category \(j\) and \(Y_{itj}=0\) otherwise. Denote by
\(\mathbf{x}_{it}\) the covariates vector associated with \(Y_{it}\),
and let
\(\mathbf x_{i}=(\mathbf x^{\top}_{i1},\ldots,\mathbf x^{\top}_{iT})^{\top}\)
be the covariates matrix for subject \(i\). Define
\(\pi_{itj}= \E(Y_{itj}|\mathbf x_i)=\Prob(Y_{itj}=1| \mathbf x_i)=\Prob(Y_{it}=j| \mathbf x_i)\)
as the probability of the response category \(j\) for subject \(i\) time
\(t\), and let
\(\boldsymbol \pi_{i}=(\boldsymbol \pi^{\top}_{i1},\ldots,\boldsymbol \pi^{\top}_{iT})^{\top}\)
be the mean vector of \(\mathbf Y_i\), where
\(\boldsymbol{\pi}_{it} = (\pi_{it1},\ldots,\pi_{it(J-1)})^{\top}\). It
follows from the above that \(Y_{itJ}=1-\sum_{j=1}^{J-1} Y_{itj}\) and
\(\pi_{itJ}=1-\sum_{j=1}^{J-1} \pi_{itj}\).

\hypertarget{marginal-models-for-correlated-multinomial-responses}{%
\subsection{Marginal models for correlated multinomial
responses}\label{marginal-models-for-correlated-multinomial-responses}}

The choice of the marginal model depends on the nature of the response
scale. For ordinal multinomial responses, the family of cumulative link
models \begin{equation}
F^{-1}\left[\Prob(Y_{it}\leq j|\mathbf x_i)\right]=\beta_{j0}+ {\boldsymbol \beta}_{\ast}^{\top} \mathbf{x}_{it}
\label{ABMCLM}
\end{equation} or the adjacent categories logit model\\
\begin{equation}
\log\left(\frac{\pi_{itj}}{\pi_{it(j+1)}} \right)=\beta_{j0}+ {\boldsymbol \beta}_{\ast}^{\top} \mathbf{x}_{it}
\label{ABMACLM}
\end{equation} can be used, where \(F\) is the cumulative distribution
function of a continuous distribution and
\(\{\beta_{j0}:j=1,\ldots,J-1\}\) are the category specific intercepts.
For nominal multinomial responses, the baseline category logit model
\begin{equation}
\log\left(\frac{\pi_{itj}}{\pi_{itJ}}\right)=\beta_{j0}+{\boldsymbol {\beta}}_{j}^{\top} \mathbf{x}_{it}
\label{ABMBCLM}
\end{equation} can be used, where \(\boldsymbol {\beta}_{j}\) is the
\(j\)-th category specific parameter vector.

It is worth mentioning that the linear predictor differs in the above
marginal models. First, the category specific intercepts need to satisfy
a monotonicity condition
\(\beta_{10}\leq\beta_{20}\leq \ldots \leq \beta_{(J-1)0}\) only when
the family of cumulative link models in (\ref{ABMCLM}) is employed.
Second, the regression parameter coefficients of the covariates
\(\mathbf x_{it}\) are category specific only in the marginal baseline
category logit model (\ref{ABMBCLM}) and not in the ordinal marginal
models (\ref{ABMCLM}) and (\ref{ABMACLM}).

\hypertarget{estimation-of-the-marginal-regression-parameter-vector}{%
\subsection{Estimation of the marginal regression parameter
vector}\label{estimation-of-the-marginal-regression-parameter-vector}}

To unify the notation, let \(\boldsymbol \beta\) be the \(p\)--variate
parameter vector that includes all the regression parameters in
(\ref{ABMCLM}), (\ref{ABMACLM}) or (\ref{ABMBCLM}). To obtain
\(\boldsymbol {\widehat \beta_G}\), a GEE estimator of
\(\boldsymbol \beta\), \citet{Touloumis2012} solved the estimating
equations \begin{equation}
\mathbf{U}(\boldsymbol \beta,\widehat{\boldsymbol \alpha})=\frac{1}{N}\sum_{i=1}^N \mathbf{D}_i \mathbf V^{-1}_{i} (\mathbf {Y}_i-\boldsymbol{\pi}_i)=\mathbf{0}
\label{EEbeta}
\end{equation} where
\(\mathbf{D}_i=\partial \boldsymbol{\pi}_i/\partial \boldsymbol{\beta}\)
and \(\mathbf V_i\) is a \(T(J-1) \times T(J-1)\) ``weight'' matrix that
depends on \(\boldsymbol \beta\) and on
\(\widehat{\boldsymbol \alpha}\), an estimate of the ``nuisance''
parameter vector \(\boldsymbol \alpha\) defined formally in Section
\protect\hyperlink{Alpha}{2.3}. Succinctly, \(\mathbf V_i\) is a block
matrix that mimics the form of \(\COV(\mathbf{Y}_i|\mathbf x_i)\), the
true covariance matrix for subject \(i\). The \(t\)-th diagonal matrix
of \(\mathbf V_i\) is the covariance matrix of \(Y_{it}\) determined by
the marginal model. The \((t,t^{\prime})\)-th off-diagonal block matrix
describes the marginal pseudo-association of
\((Y_{it},Y_{it^{\prime}})\), which is a function of the marginal model
and of the pseudo-probabilities
\(\{\Prob(Y_{it}=j,Y_{it^{\prime}}=j^{\prime}|\mathbf x_i):j,j^{\prime}=1,\ldots,J-1\}\)
calculated based on
\((\widehat{\boldsymbol \alpha},\boldsymbol \beta)\). We should
emphasize that \(\mathbf V_i\) is a ``weight'' matrix because
\(\boldsymbol \alpha\) is defined as a ``nuisance'' parameter vector and
it is unlikely to describe a valid ``working'' assumption about the
association structure for all subjects.

\hypertarget{Alpha}{%
\subsection{Estimation of the nuisance parameter vector and of the
weight matrix}\label{Alpha}}

Order the \(L=T(T-1)/2\) time-pairs with the rightmost element of the
pair most rapidly varying as \((1,2),(1,3),\ldots,(T-1,T)\), and let
\(G\) be the group variable with levels the \(L\) ordered pairs. For
each time-pair \((t,t^{\prime})\), ignore the covariates and
cross-classify the responses across subjects to form an \(J \times J\)
contingency table such that the row totals correspond to the observed
totals at time \(t\) and the column totals to the observed totals at
time \(t^{\prime}\), and let \(\theta_{tjt^{\prime}j^{\prime}}\) be the
local odds ratio at the cutpoint \((j,j^{\prime})\) based on the
expected frequencies
\(\{f_{tjt^{\prime}j^{\prime}}:j,j^{\prime}=1,\ldots,J\}\). For
notational reasons, let \(A\) and \(B\) be the row and column variable
respectively. Assuming a Poisson sampling scheme to the \(L\) sets of
\(J \times J\) contingency tables, fit the RC-G(1) type model
\citep{Becker1989a} \begin{equation}
\log f_{tjt^{\prime}j^{\prime}}=\lambda+\lambda^{A}_{j}+\lambda^{B}_{j^{\prime}}+\lambda^{G}_{(t,t^{\prime})}+\lambda^{AG}_{j(t,t^{\prime})}+\lambda^{BG}_{j^{\prime}(t,t^{\prime})}+\phi^{(t,t^{\prime})}\mu^{(t,t^{\prime})}_j \mu^{(t,t^{\prime})}_{j^{\prime}}, 
\label{RCGmodel}
\end{equation} where \(\{\mu^{(t,t^{\prime})}_{j}:j=1,\ldots,J\}\) are
the score parameters for the \(J\) response categories at the time-pair
\((t,t^{\prime})\). After imposing identifiability constraints on the
regression parameters in (\ref{RCGmodel}), the log local odds ratios
structure is given by \begin{equation}
\log \theta_{tjt^{\prime}j^{\prime}}=\phi^{(t,t^{\prime})}\left(\mu^{(t,t^{\prime})}_{j}-\mu^{(t,t^{\prime})}_{j+1}\right)\left(\mu^{(t,t^{\prime})}_{j^{\prime}}-\mu^{(t,t^{\prime})}_{j^{\prime}+1}\right).
\label{RCstructure2}
\end{equation} At each time-pair, (\ref{RCstructure2}) summarizes the
local odds ratios structure in terms of the \(J\) score parameters and
the intrinsic parameter \(\phi^{(t,t^{\prime})}\) that measures the
average association of the marginalized contingency table. Since the
score parameters do not need to be fixed or monotonic, the local odds
ratios structure is applicable to both nominal and ordinal multinomial
responses.

\citet{Touloumis2012} defined \(\boldsymbol \alpha\) as the parameter
vector that contains the marginalized local odds ratios structure
\[\alpha=\left(\theta_{1121},\ldots,\theta_{1(J-1)2(J-1)},\ldots,\theta_{(T-1)1T1},\ldots,\theta_{(T-1)(J-1)T(J-1)}\right)^{\top}\]
where \(\theta_{tjt^{\prime}j^{\prime}}\) satisfy (\ref{RCstructure2}).
To increase the parsimony of the local odds ratios structures for
ordinal responses, they proposed to use common unit-spaced score
parameters \(\left(\mu^{(t,t^{\prime})}_{j}=j\right)\) and/or common
intrinsic parameters \(\left(\phi^{(t,t^{\prime})}=\phi\right)\) across
time-pairs. For a nominal response scale, they proposed to apply a
homogeneity constraint on the score parameters
\(\left(\mu^{(t,t^{\prime})}_{j}=\mu_{j}\right)\) and use common
intrinsic parameters across time-pairs. To estimate
\(\boldsymbol \alpha\) maximum likelihood methods are involved by
treating the \(L\) marginalized contingency tables as independent.
Technical details and justification about this estimation procedure can
be found in \citet{Touloumis2011a} and \citet{Touloumis2012}.

Conditional on the estimated marginalized local odds ratios structure
\(\widehat{\boldsymbol \alpha}\) and the marginal model specification at
times \(t\) and \(t^{\prime}\),
\(\{\Prob(Y_{it}=j,Y_{it^{\prime}}=j^{\prime}|\mathbf x_i):t<t^{\prime},j,j^{\prime}=1,\ldots,J-1\}\)
are obtained as the unique solution of the iterative proportional
fitting (IPF) procedure \citep{Deming1940}. Hence, \(\mathbf V_i\) can
be readily calculated and the estimating equations in (\ref{EEbeta}) can
be solved with respect to \(\boldsymbol \beta\).

\hypertarget{asymptotic-properties-of-the-gee-estimator}{%
\subsection{Asymptotic properties of the GEE
estimator}\label{asymptotic-properties-of-the-gee-estimator}}

Given \(\widehat{\boldsymbol \alpha}\), inference about
\(\boldsymbol \beta\) is based on the fact that
\(\sqrt{N}(\boldsymbol {\widehat\beta}_G-\boldsymbol \beta)\sim \mathcal{N}(\mathbf 0,\boldsymbol {\Sigma})\)
asymptotically, where \begin{equation}
\boldsymbol {\Sigma}=\lim_{N\to\infty} N \boldsymbol {\Sigma}_0^{-1} \boldsymbol {\Sigma}_1 \boldsymbol {\Sigma}_0^{-1},
\label{RobustCovariance}
\end{equation}
\(\boldsymbol {\Sigma}_0=\sum_{i=1}^N \mathbf{D}_i^{\top} \mathbf {V}^{-1}_{i} \mathbf{D}_i\)
and
\(\boldsymbol {\Sigma}_1=\sum_{i=1}^N \mathbf{D}_i^{\top} \mathbf {V}^{-1}_{i} \COV(\mathbf{Y}_i|\mathbf x_i) \mathbf {V}^{-1}_{i} \mathbf{D}_i\).
For finite sample sizes, \(\widehat{\boldsymbol {\Sigma}}\) is estimated
by ignoring the limit in (\ref{RobustCovariance}) and replacing
\(\boldsymbol \beta\) with \(\boldsymbol {\widehat \beta}_G\) and
\(\COV(\mathbf{Y}_i|\mathbf x_i)\) with
\((\mathbf {Y}_i-\widehat{\boldsymbol{\pi}}_i)(\mathbf {Y}_i-\widehat{\boldsymbol{\pi}}_i)^{\top}\)
in \(\boldsymbol {\Sigma}_0\) and \(\boldsymbol {\Sigma}_1\). In the
literature, \(\widehat{\boldsymbol {\Sigma}}/N\) is often termed as
``sandwich'' or ``robust'' covariance matrix of
\(\boldsymbol {\widehat \beta}_G\).

\hypertarget{Description1}{%
\section{Description of core functions}\label{Description1}}

We describe the arguments of the functions \code{nomLORgee} and
\code{ordLORgee}, focusing on the marginal model specification
(\code{formula}, \code{link}), data representation (\code{id},
\code{repeated}, \code{data}) and local odds ratios structure
specification (\code{LORstr}, \code{LORterm}, \code{homogeneous},
\code{restricted}). For completeness' sake, we also present
computational related arguments (\code{LORem}, \code{add},
\code{bstart}, \code{LORgee.control}, \code{ipfp.control}, \code{IM}).
The two core functions share the same arguments, except \code{link} and
\code{restricted} which are available only in \code{ordLORgee}, and they
both create an object of the class \code{LORgee} which admits
\code{summary}, \code{coef}, \code{update}, \code{vcov}, \code{confint}
and \code{residuals} methods.

\hypertarget{marginal-model-specification}{%
\subsection{Marginal model
specification}\label{marginal-model-specification}}

For ordinal multinomial responses, the \code{link} argument in the
function \code{ordLORgee} specifies which of the marginal models
(\ref{ABMCLM}) or (\ref{ABMACLM}) will be fitted. The options
\code{"logit"}, \code{"probit"}, \code{"cauchit"} or \code{"cloglog"}
indicate the corresponding cumulative distribution function \(F\) in the
cumulative link model (\ref{ABMCLM}), while the option \code{"acl"}
implies that the adjacent categories logit model (\ref{ABMACLM}) is
selected. For nominal multinomial responses, the function
\code{nomLORgee} fits the baseline category logit model (\ref{ABMBCLM}),
and hence the \code{link} argument is not offered.

The \code{formula} (\code{= response ~ covariates}) argument identifies
the multinomial response variable (\code{response}) and specifies the
form of the linear predictor (\code{covariates}), assuming that this
includes an intercept term. If required, the \(J>2\) observed response
categories are sorted in an ascending order and then mapped onto
\(\{1,2,\ldots,J\}\). To account for a covariate \code{x} with a
constrained parameter coefficient fixed to 1 in the linear predictor,
the term \code{offset(x)} must be inserted on the right hand side of
\code{formula}.

\hypertarget{data-representation}{%
\subsection{Data representation}\label{data-representation}}

The \code{id} argument identifies the \(N\) subjects by assigning a
unique label to each subject. If required, the observed \code{id} labels
are sorted in an ascending order and then relabeled as \(1,\ldots,N\),
respectively.

The \code{repeated} argument identifies the times at which the
multinomial responses are recorded by treating the \(T\) unique observed
times in the same manner as in \code{id}. The purpose of \code{repeated}
is dual: To identify the \(T\) distinct time points and to construct the
full marginalized contingency table for each time-pair by aggregating
the relevant/available responses across subjects. The \code{repeated}
argument is optional and it can be safely ignored in balanced designs or
in unbalanced designs in which if the \(t\)-th response is missing for a
particular subject then all subsequent responses at times
\(t^{\prime}>t\) are missing for that subject. Otherwise, it is
recommended to provide the \code{repeated} argument in order to ensure
proper construction of the full marginalized contingency table. To this
end, note that if the measurement occasions are not recorded in a
numerical mode, then the user should create \code{repeated} by mapping
the \(T\) distinct measurement occasions onto the set \(\{1,\ldots,T\}\)
in such a way that the temporal order of the measurement occasions is
preserved. For example, if the measurements occasions are recorded as
``before'', ``baseline'', ``after'', then the levels for \code{repeated}
should be coded as \(1,2\) and \(3\), respectively.

The dataset is imported via the \code{data} argument in ``long'' format,
meaning that each row contains all the information provided by a subject
at a given measurement occasion. This implies that \code{data} must
include the variables specified in the mandatory arguments
\code{formula} and \code{id}, as well as the optional argument
\code{repeated} when this is specified by the user. If no \code{data} is
provided then the above variables are extracted from the
\code{environment} that \code{nomLORgee} and \code{ordLORgee} are
called. Currently missing observations, identified by \code{NA} in
\code{data}, are ignored.

\hypertarget{marginalized-local-odds-ratios-structure-specification}{%
\subsection{Marginalized local odds ratios structure
specification}\label{marginalized-local-odds-ratios-structure-specification}}

The marginalized local odds ratios is specified via the \code{LORstr}
argument. Table \ref{tab:LOR} displays the structures proposed by
\citet{Touloumis2012}. Currently the default option is the time
exchangeability structure (\code{"time.exch"}) in \code{nomLORgee} and
the category exchangeability (\code{"category.exch"}) structure in
\code{ordLORgee}. The uniform (\code{"uniform"}) and category
exchangeability structures are not allowed in \code{nomLORgee} because
given unit-spaced parameter scores are not meaningful for nominal
response categories.

The user can also fit the independence ``working'' model
(\code{LORstr = "independence"}) or even provide the local odds ratios
structure (\code{LORstr = "fixed"}) using the \code{LORterm} argument.
In this case, an \(L \times J^2\) matrix must be constructed such that
the \(g\)-th row contains the vectorized form of a probability table
that satisfies the desired local odds ratios structure at the time-pair
corresponding to the \(g\)-th level of \(G\).

\citet{Touloumis2011a} discussed two further versions of the
\code{"time.exch"} and the RC (\code{"RC"}) structures based on using:
i) Heterogeneous score parameters (\code{homogeneous = FALSE}) at each
time-pair, and/or ii) monotone score parameters
(\code{restricted = TRUE}), an option applicable only for ordinal
response categories. However, it is sensible to employ these additional
options only when the local odds ratios structures in Table
\ref{tab:LOR} do not seem adequate.

It is important to mention that the user must provide only the arguments
required for the specified local odds ratios structure. For example, the
arguments \code{homogeneous}, \code{restricted} and \code{LORterm} are
ignored when \code{LORstr = "uniform"}.

\begin{table}
\centering
\begin{tabular}{cccccl}
\hline
\hline
$\log \theta_{tjt^{\prime}j^{\prime}}$ & \code{LORstr}   & Functions & Parameters \\
\hline
$\phi$  & \code{"uniform"} & \code{ordLORgee} & 1\\
$\phi^{(t,t^{\prime})}$  &\code{"category.exch"} & \code{ordLORgee}  &L\\
$\phi \left(\mu_{j}-\mu_{j+1}\right)\left(\mu_{j^{\prime}}-\mu_{j^{\prime}+1}\right)$ & \code{"time.exch"}  & Both  & $J-1$ \\
$\phi^{(t,t^{\prime})}\left(\mu^{(t,t^{\prime})}_{j}-\mu^{(t,t^{\prime})}_{j+1}\right)\left(\mu^{(t,t^{\prime})}_{j^{\prime}}-\mu^{(t,t^{\prime})}_{j^{\prime}+1}\right)$ & \code{"RC"}    & Both & $L(J-1)$ \\
\hline
\end{tabular}
\caption{The main options for the marginalized local odds ratios structures in \pkg{multgee}.}
\label{tab:LOR}
\end{table}

\hypertarget{computational-details}{%
\subsection{Computational details}\label{computational-details}}

The default estimation procedure for the marginalized local odds ratios
structure is to fit model (\ref{RCGmodel}) to the full marginalized
contingency table (\code{LORem = "3way"}) after imposing the desired
restrictions on the intrinsic and the score parameters.
\citet{Touloumis2011a} noticed that the estimated local odds ratios
structure under model (\ref{RCGmodel}) is identical to that obtained by
fitting independently a row and columns (RC) effect model
\citep{Goodman1985} with homogeneous score parameters to each of the
\(L\) contingency tables. Motivated by this, an alternative estimation
procedure (\code{LORem = "2way"}) for estimating the structures
\code{"uniform"} and \code{"time.exch"} was proposed. In particular, one
can estimate the single parameter of the \code{"uniform"} structure as
the average of the \(L\) intrinsic parameters \(\phi^{(t,t^{\prime})}\)
obtained by fitting the linear-by-linear association model
\citep{Agresti2002} independently to each of the \(L\) marginalized
contingency tables. For the \code{"time.exch"} structure, one can fit
\(L\) RC effects models with homogeneous
(\code{homogeneous = TRUE})/heterogeneous (\code{homogeneous = FALSE})
score parameters and then estimate the log local odds ratio at each
cutpoint \((j,j^{\prime})\) by averaging
\(\log \hat{\theta}_{tjt^{\prime}j^{\prime}}\) for \(t<t^{\prime}\).
Regardless of the value of \code{LORem}, the appropriate model for
counts is fitted via the function \code{gnm} of the \proglang{R} package
\pkg{gnm} \citep{Turner2012}.

In the presence of zero observed counts, a small positive constant can
be added (\code{add}) at each cell of the marginalized contingency table
to ensure the existence of \(\widehat{\boldsymbol \alpha}\). We
conjecture that a constant of the magnitude \(10^{-4}\) will serve this
purpose without affecting the strength of the association structure.

A Fisher scoring algorithm is employed to solve the estimating equations
(\ref{EEbeta}) as in \citet{Lipsitz1994}. The only difference is that
now \(\widehat{\boldsymbol{\alpha}}\) is not updated. The default way to
obtain the initial value for \(\boldsymbol \beta\) is via the function
\code{vglm} of the R package \pkg{VGAM} \citep{Yee2010}. Alternatively,
the initial value can be provided by the user (\code{bstart}). The
Fisher scoring algorithm converges when the elementwise maximum relative
change in two consecutive estimates of \(\boldsymbol \beta\) is less
than or equal to a predefined positive constant \(\epsilon\). The
\code{control} argument controls the related iterative procedure
variables and printing options. The default maximum number of iterations
is \(15\) and the default tolerance is \(\epsilon=0.001\).

Recall that calculation of the ``weight'' matrix \(\mathbf V_i\) at
given values of \((\boldsymbol \beta,\boldsymbol \alpha)\) relies on the
IPF procedure. The \code{ipfp.ctrl} argument controls the related
variables. The convergence criterion is the maximum of the absolute
difference between the fitted and the target row and column marginals.
By default, the tolerance of the IPF procedure is \(10^{-6}\) with a
maximal number of iterations equal to 200.

The \code{IM} argument defines which of the \proglang{R} functions
\code{solve}, \code{qr.solve} or \code{cholesky} will be used to invert
matrices in the Fisher scoring algorithm.

\hypertarget{Description2}{%
\section{Description of utility functions}\label{Description2}}

The function \code{waldts} performs a goodness-of-fit test for two
nested GEE models based on a Wald test statistic. Let \(\mathrm{M_0}\)
and \(\mathrm{M_1}\) be two nested GEE models with marginal regression
parameter vectors \(\boldsymbol \beta_0\) and
\(\boldsymbol \beta_1=(\boldsymbol \beta_0^{\top},\boldsymbol \beta^{\top}_q)^{\top}\),
respectively. Define a matrix \(\mathbf C\) such that
\(\mathbf C \boldsymbol \beta_1=\boldsymbol \beta_q\). Here \(q\) equals
the rank of \(\mathbf C\) and the dimension of \(\boldsymbol \beta_q\).
The hypothesis\\
\[H_0: \boldsymbol \beta_q=0 \text{ vs } H_1: \boldsymbol \beta_q \neq 0\]
tests the goodness-of-fit of \(\mathrm{M_0}\) versus \(\mathrm{M_1}\).
Based on a Wald type approach, \(H_0\) is rejected at \(\alpha \%\)
significance level, if
\((\mathbf C \widehat{\boldsymbol \beta})^{\top} (N\mathbf C \widehat{\boldsymbol \Sigma}\mathbf C^{\top})^{-1}(\mathbf C \widehat{\boldsymbol \beta}) \geq X_{q}(\alpha)\),
where \(\widehat{\boldsymbol \beta}\) and
\(\widehat{\boldsymbol \Sigma}\) are estimated under model
\(\mathrm{M_1}\) and \(X_{q}(\alpha)\) denotes the \(\alpha\) upper
quantile of a chi-square distribution with \(q\) degrees of freedom.

The function \code{confint} calculates the Wald-type confidence
intervals for the regression parameters of the marginal model under
consideration. The \code{method} argument specifies whether the standard
errors of the regression parameters estimates will be derived from the
``sandwich'' (\code{method = "robust"}) or the ``model-based''
(\code{method = "naive"}) covariance matrix. The default is to use the
``sandwich'' covariance matrix.

\citet{Touloumis2012} suggested to select the local odds ratios
structure by inspecting the range of the \(L\) estimated intrinsic
parameters under the \code{"category.exch"} structure for ordinal
responses, or under the \code{"RC"} structure for nominal responses. If
the estimated intrinsic parameters do not differ much, then the
underlying marginalized local odds ratios structure is likely nearly
exchangeable across time-pairs. In this case, the simple structures
\code{"uniform"} or \code{"time.exch"} should be preferred because they
tend to be as efficient as the more complicated ones. The function
\code{intrinsic.pars} gives the estimated intrinsic parameter of each
time-pair.

The function \code{vcov} returns the ``sandwich''
(\code{method = "robust"}) or the ``model-based''
(\code{method = "naive"}) covariance matrix. The default is to use the
``sandwich'' covariance matrix. It is used internally to calculate
confidence intervals for the regression parameters and to compare nested
models.

The single-argument function \code{matrixLOR} creates a two-way
probability table that satisfies a desired local odds ratios structure.
This function aims to ease the construction of the \code{LORterm}
argument in the core functions \code{nomLORgee} and \code{ordLORgee}.

\hypertarget{Example}{%
\section{Example}\label{Example}}

To illustrate the main features of the package \pkg{multgee}, we follow
the GEE analysis performed in \citet{Touloumis2012}. The data came from
a randomized clinical trial \citep{Lipsitz1994} that aimed to evaluate
the effectiveness of the drug Auranofin versus the placebo therapy for
the treatment of rheumatoid arthritis. The five-level (1=poor, \ldots,
5=very good) ordinal multinomial response variable was the
self-assessment of rheumatoid arthritis recorded at one (\(t=1\)), three
(\(t=2\)) and five (\(t=3\)) follow-up months. To acknowledge the
ordinal response scale, the marginal cumulative logit model
\begin{align}
\log \left(\frac{\Prob(Y_{it}\leq j|\mathbf x_i)}{1-\Prob(Y_{it}\leq j|\mathbf x_i)}\right)&=\beta_{j0}+\beta_1 I(time_i=3)+\beta_2 I(time_i=5) +\beta_3 trt_i   \nonumber\\ 
                                                     &+\beta_4 I(b_i=2)+\beta_5 I(b_i=3)+\beta_6 I(b_i=4)+\beta_7 I(b_i=5).
\label{MarginalModelData}
\end{align} was fitted, where \(i=1,\ldots,301\), \(t=1,2,3\),
\(j=1,2,3,4\) and \(I(A)\) is the indicator function for the event
\(A\). Here \(\mathbf x_i\) denotes the covariates matrix for subject
\(i\) that includes the self-assessment of rheumatoid arthritis at the
baseline (\(b_i\)), the treatment variable (\(trt_i\)), coded as \((1)\)
for the placebo group and \((2)\) for the drug group, and the follow-up
time recorded in months (\(time_i\)).

The GEE analysis is performed in two steps. First, we select the
marginalized local odds ratios structure by estimating the intrinsic
parameters under the \code{"category.exch"} structure

\begin{CodeChunk}

\begin{CodeInput}
R> library("multgee")
R> data("arthritis")
R> head(arthritis)
\end{CodeInput}

\begin{CodeOutput}
  id y sex age trt baseline time
1  1 4   2  54   2        2    1
2  1 5   2  54   2        2    3
3  1 5   2  54   2        2    5
4  2 4   1  41   1        3    1
5  2 4   1  41   1        3    3
6  2 4   1  41   1        3    5
\end{CodeOutput}

\begin{CodeInput}
R> intrinsic.pars(y = y, data = arthritis, id = id, repeated = time,
R+                   rscale = "ordinal")
\end{CodeInput}

\begin{CodeOutput}
[1] 0.6517843 0.9097341 0.9022272
\end{CodeOutput}
\end{CodeChunk}

The range of the estimated intrinsic parameters is small
(\(\approx 0.26\)) which suggests that the underlying marginalized
association pattern is nearly constant across time-pairs. Thus we expect
the \code{"uniform"} structure to capture adequately the underlying
correlation pattern. Note that we passed the time variable to the
\code{repeated} argument because this numerical variable indicates the
measurement occasion at which each observation was recorded.

Now we fit the cumulative logit model (\ref{MarginalModelData}) under
the \code{"uniform"} via the function \code{ordLORgee}

\begin{CodeChunk}

\begin{CodeInput}
R>  fit <- ordLORgee(formula = y ~ factor(time) + factor(trt) + factor(baseline),
R+         link = "logit", id = id, repeated = time, data = arthritis,
R+         LORstr = "uniform")
R>  summary(fit)
\end{CodeInput}

\begin{CodeOutput}
GEE FOR ORDINAL MULTINOMIAL RESPONSES 
version 1.6.0 modified 2017-07-10 

Link : Cumulative logit 

Local Odds Ratios:
Structure:         uniform
Model:             3way

call:
ordLORgee(formula = y ~ factor(time) + factor(trt) + factor(baseline), 
    data = arthritis, id = id, repeated = time, link = "logit", 
    LORstr = "uniform")

Summary of residuals:
      Min.    1st Qu.     Median       Mean    3rd Qu.       Max. 
-0.5161111 -0.2399227 -0.0749670  0.0000219 -0.0066995  0.9932912 

Number of Iterations: 5 

Coefficients:
                  Estimate   san.se   san.z Pr(>|san.z|)    
beta10            -1.84315  0.38929 -4.7346      < 2e-16 ***
beta20             0.26692  0.35013  0.7624      0.44585    
beta30             2.23132  0.36625  6.0924      < 2e-16 ***
beta40             4.52542  0.42123 10.7434      < 2e-16 ***
factor(time)3      0.00140  0.12183  0.0115      0.99080    
factor(time)5     -0.36172  0.11395 -3.1743      0.00150 ** 
factor(trt)2      -0.51212  0.16799 -3.0486      0.00230 ** 
factor(baseline)2 -0.66963  0.38036 -1.7605      0.07832 .  
factor(baseline)3 -1.26070  0.35252 -3.5763      0.00035 ***
factor(baseline)4 -2.64373  0.41282 -6.4041      < 2e-16 ***
factor(baseline)5 -3.96613  0.53164 -7.4602      < 2e-16 ***
---
Signif. codes:  0 '***' 0.001 '**' 0.01 '*' 0.05 '.' 0.1 ' ' 1

Local Odds Ratios Estimates:
       [,1]  [,2]  [,3]  [,4]  [,5]  [,6]  [,7]  [,8]  [,9] [,10] [,11] [,12]
 [1,] 0.000 0.000 0.000 0.000 2.257 2.257 2.257 2.257 2.257 2.257 2.257 2.257
 [2,] 0.000 0.000 0.000 0.000 2.257 2.257 2.257 2.257 2.257 2.257 2.257 2.257
 [3,] 0.000 0.000 0.000 0.000 2.257 2.257 2.257 2.257 2.257 2.257 2.257 2.257
 [4,] 0.000 0.000 0.000 0.000 2.257 2.257 2.257 2.257 2.257 2.257 2.257 2.257
 [5,] 2.257 2.257 2.257 2.257 0.000 0.000 0.000 0.000 2.257 2.257 2.257 2.257
 [6,] 2.257 2.257 2.257 2.257 0.000 0.000 0.000 0.000 2.257 2.257 2.257 2.257
 [7,] 2.257 2.257 2.257 2.257 0.000 0.000 0.000 0.000 2.257 2.257 2.257 2.257
 [8,] 2.257 2.257 2.257 2.257 0.000 0.000 0.000 0.000 2.257 2.257 2.257 2.257
 [9,] 2.257 2.257 2.257 2.257 2.257 2.257 2.257 2.257 0.000 0.000 0.000 0.000
[10,] 2.257 2.257 2.257 2.257 2.257 2.257 2.257 2.257 0.000 0.000 0.000 0.000
[11,] 2.257 2.257 2.257 2.257 2.257 2.257 2.257 2.257 0.000 0.000 0.000 0.000
[12,] 2.257 2.257 2.257 2.257 2.257 2.257 2.257 2.257 0.000 0.000 0.000 0.000

p-value of Null model: < 0.0001 
\end{CodeOutput}
\end{CodeChunk}

The \code{summary} method summarizes the fit of the GEE model including
the GEE estimates, their estimated standard errors based on the
``sandwich'' covariance matrix and the \(p\)--values from testing the
statistical significance of each regression parameter in
(\ref{MarginalModelData}). The estimated marginalized local odds ratios
structure can be found in a symmetric \(T(J-1) \times T(J-1)\) block
matrix written symbolically as \[\begin{bmatrix}
\begin{array}{cccc}
\mathbf 0                               &\boldsymbol{\Theta}_{12}                 &\ldots  &\boldsymbol{\Theta}_{1T} \\
\boldsymbol{\Theta}_{21}       &\mathbf 0                                &\ldots  &\boldsymbol{\Theta}_{2T} \\
\ldots                                  &\ldots                                   &\ddots  & \ldots          \\
\boldsymbol{\Theta}_{T1}       &\boldsymbol{\Theta}_{T2}        &\ldots  &\mathbf 0
\end{array}
\end{bmatrix}.\] Each block denotes an \((J-1) \times (J-1)\) matrix.
The (\(j,j^{\prime}\))-th element of the off-diagonal block
\(\boldsymbol{\Theta}_{tt^{\prime}}\) represents the estimate of
\(\theta_{tjt^{\prime}j^{\prime}}\). Based on the properties of the
local odds ratios it is easy to see that
\(\boldsymbol{\Theta}_{tt^{\prime}}=\boldsymbol{\Theta}^{\top}_{t^{\prime}t}\)
for \(t<t^{\prime}\). Finally, the diagonal blocks are zero to reflect
the fact that no local odds ratios are estimated when \(t=t^{\prime}\).
In our example, \(J=5\) and thus each block is a \(4 \times 4\) matrix.
Since the \code{uniform} structure is selected, all local odds ratios
are equal and estimated as \(2.257\). Finally,
\code{pvalue of Null model} corresponds to the \(p\)--value of testing
the hypothesis that no covariate is significant, i.e.,
\(\beta_1=\beta_2=\beta_3=\beta_4 =\beta_5=\beta_6=\beta_7=0\), based on
a Wald test statistic.

The goodness-of-fit of model (\ref{MarginalModelData}) can be tested by
comparing it to a marginal cumulative logit model that additionally
contains the age and gender main effects in the linear predictor

\begin{CodeChunk}

\begin{CodeInput}
R> fit1 <- update(fit, formula = ~. + factor(sex) + age)
R> waldts(fit, fit1)
\end{CodeInput}

\begin{CodeOutput}
Goodness of Fit based on the Wald test 

Model under H_0: y ~ factor(time) + factor(trt) + factor(baseline)
Model under H_1: y ~ factor(time) + factor(trt) + factor(baseline) + factor(sex) + 
    age

Wald Statistic = 3.955443, df = 2, p-value = 0.1384
\end{CodeOutput}
\end{CodeChunk}

The Wald confidence intervals for the regression parameters of the
reduced model are:

\begin{CodeChunk}

\begin{CodeInput}
R> confint(fit)
\end{CodeInput}

\begin{CodeOutput}
                       2.5 %      97.5 %
beta10            -2.6061497 -1.08015721
beta20            -0.4193163  0.95314883
beta30             1.5134867  2.94915281
beta40             3.6998267  5.35101277
factor(time)3     -0.2373805  0.24018865
factor(time)5     -0.5850535 -0.13837735
factor(trt)2      -0.8413778 -0.18287050
factor(baseline)2 -1.4151270  0.07587068
factor(baseline)3 -1.9516257 -0.56978096
factor(baseline)4 -3.4528305 -1.83462340
factor(baseline)5 -5.0081187 -2.92413445
\end{CodeOutput}
\end{CodeChunk}

\hypertarget{Summary}{%
\section{Summary and practical guidelines}\label{Summary}}

We described the \proglang{R} package \pkg{multgee} which implements the
local odds ratios GEE approach \citep{Touloumis2012} for correlated
multinomial responses. Unlike existing GEE softwares, \pkg{multgee}
allows GEE models for ordinal (\code{ordLORgee}) and nominal
(\code{nomLORgee}) responses. The available local odds ratios structures
(\code{LORstr}) in each function respect the nature of the response
scale to prevent usage of ordinal local odds ratios structures (e.g.,
\code{"uniform"}) in \code{nomLORgee}. The fitted GEE model is
summarized via the \code{summary} method while the estimated regression
coefficient can be retrieved via the \code{coef} method. The statistical
significance of the regression parameters can be assessed via the
function \code{waldts}. Confidence intervals of a desired marginal model
can be obtained by the function \code{confint}. A similar strategy to
that presented in Section \protect\hyperlink{Example}{5}, can be adopted
to analyze GEE models for correlated nominal multinomial responses.

From a practical point of view, we recommend the use of the
\code{"uniform"} structure for ordinal responses and the
\code{"time.exch"} structure for nominal especially when the range of
the estimated intrinsic parameters (\code{intrinsic.pars}) is small.
Based on our experience, some convergence problems might occur as the
complexity of the local odds ratios structure increases and/or if the
marginalized contingency tables are very sparse. Two possible solutions
are either to adopt a simpler local odds ratios structure or to increase
slightly the value of the constant added to the marginalized contingency
tables (\code{add}). However, we believe that users should refrain from
using the independence ``working'' model unless the aforementioned
strategies fail to remedy the convergence problems. To decide on the
form of the linear predictor, variable selection model procedures could
be incorporated using the function \code{waldts}. Finally, confidence
intervals could be obtained by using the function \code{confint}.

In future versions of \pkg{multgee}, we plan to permit time-dependent
intercepts in the marginal models, to increase the range of the marginal
models, by including, for example, the family of continuation-ratio
models for ordinal responses, and to offer a function for assessing the
proportional odds assumption in models (\ref{ABMCLM}) and
(\ref{ABMACLM}).

\bibliography{References.bib}


\end{document}
